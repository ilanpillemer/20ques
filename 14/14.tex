\documentclass{article}
%\usepackage[margin=2in]{geometry}
\usepackage[osf,p]{libertinus}
\usepackage{microtype}
\usepackage[pdfusetitle,hidelinks]{hyperref}

\usepackage[series={A,B,C}]{reledmac}
\usepackage{reledpar}

\usepackage{graphicx}
\usepackage{polyglossia}
\setmainlanguage{english}
\setotherlanguage{hebrew}
\gappto\captionshebrew{\renewcommand\chaptername{קאַפּיטל}}
\usepackage{metalogo}


%%linenumincrement*{1}
%%\firstlinenum*{1}
%%\setlength{\Lcolwidth}{0.44\textwidth}
%%\setlength{\Rcolwidth}{0.44\textwidth}

\begin{document}
%%\maxhnotesA{0.8\textheight}
\renewcommand{\abstractname}{\vspace{-\baselineskip}}
\title{Question 14, 20 Questions about Shoah Literature.}
\author{Transl. Ilan Pillemer}
\date{\today}

\maketitle
\abstract{
Translation Exercise week 2
}
\newline

\begin{pairs}

\begin{Rightside}

\begin{RTL}
\begin{hebrew}
\beginnumbering
\autopar
\eledsection*{
װאָס װײסן מיר װעגען די שריפֿטן פֿון ק. צעטניק און זײַן ראָמאַן ``סאַלאַמאַנדראַ''?
}

ביז דעם אײכמאַן-משפּט אין ישראל, האָט קײנער נישט געװוּסט װער איז ער, אָט דער מחבר װאָס שרײַבט זיך אונטער מיטן נאָמען קאַצעטניק צוזאַמען מיט אַן אויסװיץ-נומער,
 און ביז הײַינט איז נאָך זײַן ייִדישער ראָמאַן ``סאַלאַמאַנדער'' נישט אָפּגעדרוקט געװאָרן.
 

\endnumbering
\end{hebrew}
\end{RTL}
\end{Rightside}


\begin{Leftside}
\begin{english}
\beginnumbering
\autopar
\eledsection*{
What do we know about the works of Ka-Tzetnik\footnotemark[0]  and his novel "Salamandra"?
}
 
 Up until the Eichman Trial in Yisrael, no-one knew who he was, just that he was the author which wrote under the name 
 ``Ka-Tzetnik'' together with an Aushwitz number, and, up until today still his Yiddish novel "Salamandra" has not been published.
 
 
\endnumbering
\end{english}
\end{Leftside}

\end{pairs}
\Columns

\footnotetext[0]{
``Ka-Tzetnik'' is a word used to describe someone incarcerated in a concentration camp. So his pseudonym under 
which he published all his works post Shoah, was simply ``a concentration camp prisoner''
}
\end{document}
