\documentclass{article}
%\usepackage[margin=2in]{geometry}
\usepackage[osf,p]{libertinus}
\usepackage{microtype}
\usepackage[pdfusetitle,hidelinks]{hyperref}

\usepackage[series={A,B,C}]{reledmac}
\usepackage{reledpar}

\usepackage{graphicx}
\usepackage{polyglossia}
\setmainlanguage{english}
\setotherlanguage{hebrew}
\gappto\captionshebrew{\renewcommand\chaptername{קאַפּיטל}}
\usepackage{metalogo}


%%linenumincrement*{1}
%%\firstlinenum*{1}
%%\setlength{\Lcolwidth}{0.44\textwidth}
%%\setlength{\Rcolwidth}{0.44\textwidth}

\begin{document}
%%\maxhnotesA{0.8\textheight}
\renewcommand{\abstractname}{\vspace{-\baselineskip}}
\title{Question 14, 20 Questions about Shoah Literature.}
\author{Transl. Ilan Pillemer}
\date{\today}

\maketitle
\abstract{
Translation Exercise week 2
}
\newline

\begin{pairs}

\begin{Rightside}

\begin{RTL}
\begin{hebrew}
\beginnumbering
\autopar
\emph{
14.
װאָס װײסן מיר װעגען די שריפֿטן פֿון ק. צעטניק און זײַן ראָמאַן ``סאַלאַמאַנדראַ''?
}
\newline

ביז דעם אײכמאַן-משפּט אין ישראל, האָט קײנער נישט געװוּסט װער איז ער, אָט דער מחבר װאָס שרײַבט זיך אונטער מיטן נאָמען קאַצעטניק צוזאַמען מיט אַן אויסװיץ-נומער,
און ביז הײַנט איז נאָך זײַן ייִדישער ראָמאַן ``סאַלאַמאַנדער'' נישט אָפּגעדרוקט געװאָרן.
עס איז דאָ אַ סבֿרה אַז אַ דײַטשע איבערזעצונג פֿון ראָמאַן װעט דערשײַנען נאָך אײדער עמעצער װעט זיך פֿעדערן און אָפּדרוקן דעם אָריגינאַל אין ישראל.


די עניגמאַ ``קאַצעטניק'', בלײַבט דערװײַל אַ רעטעניש, און אַפֿילו װען מיט יאָרן צוריק זענען געמאַכט גװאָרען פּרוּװן זיך דערװיסן װעגן אים, װען און װאָס -
װײס די װעלט װעגן אים זײער װײניק.
װי עס זעט אויס, איז זײַן ראָמאַן סאַלאַמאַנדער װעגן אומקום, געװען די ערשטע פֿולע פּראָזע-שפֿונג געשריבן פֿון אַ געראַנטעװעטן שרײַבער, װעלכער איז אויך פֿאַר דער מלחמה
געװען אַ ייִדישער שרײַבער. 
געבוירן דעם 16טן מײַ 1909 אין דעם דאָרף סטאַברוּװ לעבן סאָסנאָװיק -
װי יחיאל פֿײנער - איז ער שוין פֿאַר דער מלחמה געװען אַ ייִדישער שרײַבער, אַ פֿאָרשטײער פֿון די יונגע כּוחות אין ``אַגודת-ישׂראל'' אין פּוילן 
און אַ לערער פֿון העברעיִש און תּנך.
זײַן ראָמאַן "סאַלאַמאַנדער" געדרוקט סוף 1946 אין ישׂראל אין אַ העברעיִשער איבערלעבונגען -
באַשרײַבט צום ערשטן מאָל פֿאַר דער פֿרײַער װעלט אַן אַלגעמײן בילד פֿון דער אומקום-תּקופֿה, 
באַזירט אויף קאַצעטניקס אויטענישע איבערלעבונגען אין געטאָ און אין פֿיר לאַנגערן - אוישװיץ אַרײַנגערעכנט.


װי אַ תּלמיד-חכם און אַ געװעזענער ישיבֿה-בחור פֿון הרבֿ שאַפּיראַס ישיבֿה אין לובלין, װערן ייִדשע  מקורים פֿון דער גמרא אַרײַנגעפֿלאַכטן אין זײַן ראָמאַן.
למשל, דער נאָמען פֿון ראָמאַן: "סאַלאַמאַנדער" איז מרמז אויף דעם אַז קאַצעטניק איז געפּרוּװעט אין פֿײַער פֿון די קרעמאַטאָריעם אין אוישװיץ - און 
איז אַרויס אַ לעבעדיקער. 
װעגן אים און זײַן  ראָמאַן, איז דערשינען אויף העברעיִש דאָס בוך סאַלאַמאַנדרה - מיתוס והיסטוריה בכתבי ק.צעטניק, ירושלים,  מפעל דב סדן והוצאת כרמל, 2009

\endnumbering
\end{hebrew}
\end{RTL}
\end{Rightside}


\begin{Leftside}
\begin{english}
\beginnumbering
\autopar
\emph{
14.
What do we know about the works of Ka-Tzetnik  and his novel "Salamandra"?
}
\newline 
 
 Up until the Eichman Trial in Yisrael, no-one knew who he was, just that he was the author which wrote under the name 
 ``Ka-Tzetnik''\footnoteA{``Ka-Tzetnik'' is a word used to describe someone incarcerated in a concentration camp. So his pseudonym under 
which he published all his works post Shoah, was simply ``a concentration camp prisoner''}
together with an Auschwitz number, and, up until today still his Yiddish novel "Salamandra" has not been published.
There is a chance that a German translation of the novel will be published even before someone\footnoteA{probably he is referring to 
whomever bought the Yiddish manuscript when it went up for auction in 2010. \newline
https://il.bidspirit.com/ui/lotPage/source/search/auction/7018/lot/146972/lot?lang=en }
will get up off their laurels and publish the Yiddish original in Yisrael.


 The enigma of ``Ka-Tzetnik'' remains, for the time being, a riddle; and despite attempts that have been made in 
 the previous years to learn about him - the whens and the whats,  
the world knows very little about him.
It appears to be, that his novel "Salamandra" about the Holocaust, was the first complete work of prose written by a successful-to-be writer, whom was also
before the war a Yiddish writer.
Born on the 16th May 1909 in the village of Stabrów near Sosnowiec - as Yehiel Feiner - he was already before the war a Yiddish writer, a delegate 
for the youth efforts in "Agudat-Yisrael" in Poland, and a teacher of Hebrew and Tanach. 
His novel "Salamander" published at the end of 1946 in Yisrael in a Hebrew translation -
described for the first time to the free world a general picture of the Holocaust period,
based on Katzetnik's authentic life experiences in the ghetto and in four camps - Auschwitz included.

As a talmid-chacham and a previous yeshiva-bocher of Rav Shapira's Yeshiva in Lublin, Jewish sources from the Gemara were weaved into his novel.
For example, the name of the novel: "Salamander"\footnoteA{\emph{Sanhedrin 63b} - On this page of Gemara the talmudic voices discuss how a child is saved 
from being burnt alive as a sacrifice because his mother smeared him with the blood of a Salamandra. } \footnoteA{\emph{Midrash Tanchumah} - In this midrash 
when discussing Parashat veYeshav, the midrash discusses how a Salamandra is born and lives in fire, and that if you cover your body with its blood you become
impervious to the flames.} \footnoteA{\emph{Hagiga 27a} - Here a talmud-chaham is said to be impervious to fire as just like a Salamandra's blood protects the body from the fire, all the more so
are those that cover themselves with the words of the divine.}
 alludes to that Ka-Tzetnik was tested in the fire of the crematorium in Auschwitz - 
and emerged living.
Concerning him and his novel, this book has been a published in Hebrew "Salamandra - Myth and Hisory in Katzenik's Writings", Jerusalem, Carmel Publishing House, 2009

\endnumbering
\end{english}
\end{Leftside}

\end{pairs}
\Columns


\end{document}



















































