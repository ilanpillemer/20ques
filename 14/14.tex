\documentclass{article}
%\usepackage[margin=2in]{geometry}
\usepackage[osf,p]{libertinus}
\usepackage{microtype}
\usepackage[pdfusetitle,hidelinks]{hyperref}

\usepackage[series={A,B,C}]{reledmac}
\usepackage{reledpar}

\usepackage{graphicx}
\usepackage{polyglossia}
\setmainlanguage{english}
\setotherlanguage{hebrew}
\gappto\captionshebrew{\renewcommand\chaptername{קאַפּיטל}}
\usepackage{metalogo}


%%linenumincrement*{1}
%%\firstlinenum*{1}
%%\setlength{\Lcolwidth}{0.44\textwidth}
%%\setlength{\Rcolwidth}{0.44\textwidth}

\begin{document}
%%\maxhnotesA{0.8\textheight}
\renewcommand{\abstractname}{\vspace{-\baselineskip}}
\title{Question 14, 20 Questions about Shoah Literature.}
\author{Transl. Ilan Pillemer}
\date{\today}

\maketitle
\abstract{
Translation Exercise week 2
}
\newline

\begin{pairs}

\begin{Rightside}

\begin{RTL}
\begin{hebrew}
\beginnumbering
\autopar
\emph{
14.
װאָס װײסן מיר װעגען די שריפֿטן פֿון ק. צעטניק און זײַן ראָמאַן ``סאַלאַמאַנדראַ''?
}
\newline

ביז דעם אײכמאַן-משפּט אין ישראל, האָט קײנער נישט געװוּסט װער איז ער, אָט דער מחבר װאָס שרײַבט זיך אונטער מיטן נאָמען קאַצעטניק צוזאַמען מיט אַן אויסװיץ-נומער,
און ביז הײַנט איז נאָך זײַן ייִדישער ראָמאַן ``סאַלאַמאַנדער'' נישט אָפּגעדרוקט געװאָרן.
עס איז דאָ אַ סבֿרה אַז אַ דײַטשע איבערזעצונג פֿון ראָמאַן װעט דערשײַנען נאָך אײדער עמעצער װעט זיך פֿעדערן און אָפּדרוקן דעם אָריגינאַל אין ישראל.

\endnumbering
\end{hebrew}
\end{RTL}
\end{Rightside}


\begin{Leftside}
\begin{english}
\beginnumbering
\autopar
\emph{
14.
What do we know about the works of Ka-Tzetnik  and his novel "Salamandra"?
}
\newline 
 
 Up until the Eichman Trial in Yisrael, no-one knew who he was, just that he was the author which wrote under the name 
 ``Ka-Tzetnik''\footnoteA{``Ka-Tzetnik'' is a word used to describe someone incarcerated in a concentration camp. So his pseudonym under 
which he published all his works post Shoah, was simply ``a concentration camp prisoner''}
together with an Auschwitz number, and, up until today still his Yiddish novel "Salamandra" has not been published.
There is a chance that a German translation of the novel will be published even before someone\footnoteA{probably he is referring to 
whomever bought the Yiddish manuscript when it went up for auction in 2010. \newline
https://il.bidspirit.com/ui/lotPage/source/search/auction/7018/lot/146972/lot?lang=en }
will get up off their laurels and publish the Yiddish original in Yisrael.
 
\endnumbering
\end{english}
\end{Leftside}

\end{pairs}
\Columns


\end{document}
